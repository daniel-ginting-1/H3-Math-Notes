\documentclass[a4paper]{article}

\usepackage{amssymb}
\usepackage{amsmath}
\usepackage{mathtools}

%opening
\title{Daniel's H3 Mathematics Notes}
\author{Daniel Ginting}
\date{\today}

\begin{document}
	
\maketitle

\newpage
\thispagestyle{empty}

\vspace*{\fill}
\begin{center}
	\emph{To poki.}
\end{center}
\vspace*{\fill}

\newpage

\begin{abstract}

This is my H3 Mathematics notes. This paper is nothing less than a re-formatting of Dr Jonathon Teo's H3 Slides in 2025. The contents therefore are not original to the author.

\end{abstract}




\section{Nomenclature and Notations}

\begin{itemize}
	\item $\mathbb{Z}$ denotes the set of integers.
	\item $\mathbb{Z}^+$ denotes the set of positive integers.
	\item $\mathbb{N}$ denotes the set of natural numbers.
	\item $\mathbb{Q}$ denotes the set of rational numbers.
	\item $\varnothing$ denotes the empty set.
	
\end{itemize}

\section{Mathematical Statements}

\subsection{Structure}

\subsubsection{Conditional Statements}

\subsubsection{Biconditional Statements}

\subsubsection{Existential Statements}

\subsection{Content}

A (mathematical) definition is a (true) mathematical statement that gives the precise meaning of a word
or phrase that represents some object, property or other concepts.

A theorem is a true mathematical statement that can be proven mathematically.

An axiom is a mathematical statement that does not require proof.

\section{Proofs}

A (mathematical) proof is a deductive argument for a mathematical statement, showing that the stated assumptions logically guarantee the conclusion.

\[
P \xLongrightarrow{\substack{
		\text{definitions, axioms, algebraic manipulations,} \\
		\text{and previously established facts}
}} Q
\]

\subsection{Direct Proof}

A direct proof is an approach to prove a conditional statement $P \Rightarrow Q$.

\subsection{Counter Example}

\subsection{Proof by Cases}

\subsection{Proof by Contradiction}

\subsection{Proving Existential Statements}

\subsubsection{Constructive Proof}

\subsubsection{Non-constructive Proof}

\section{Congruence Modulo}

\subsection{Properties of Congruence}

\subsection{Modular Aritmethic}

\section{Truth Table}

\section{Negation}

\section{Sequences and Series}

\section{Pigeon Hole Principle}






\end{document}
