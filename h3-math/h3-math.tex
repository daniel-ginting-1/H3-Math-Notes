\documentclass[a4paper]{article}

\usepackage{amssymb}
\usepackage{amsmath}
\usepackage{mathtools}
\usepackage{amsthm}
\usepackage{enumitem}
\usepackage[hidelinks]{hyperref}

\urlstyle{same}

\theoremstyle{definition}
\newtheorem*{example}{Example}
\newtheorem*{solution}{Solution}
\newtheorem*{definition}{Definition}
\newtheorem*{theorem}{Theorem}
\newtheorem*{exercise}{Exercise}


%opening
\title{Daniel's H3 Mathematics Notes}
\author{Daniel Ginting}
\date{\today}

\begin{document}
	
\maketitle

\newpage
\thispagestyle{empty}

\vspace*{\fill}
\begin{center}
	\emph{To poki.}
\end{center}
\vspace*{\fill}

\newpage

\begin{abstract}

This is my H3 Mathematics notes. This paper is nothing less than a re-formatting of Dr Jonathon Teo's H3 Slides in 2025. The contents therefore are not original to the author.

\end{abstract}




\section{Some Nomenclature and Notations}

The set of real numbers is denoted by $\mathbb{R}$. Unless stated otherwise, the universal set of numbers is the set of real numbers.

\begin{itemize}
	
	\item $\mathbb{Z} = \{0, \pm 1, \pm 2, \pm 3, \pm 4, \ldots\}$: the set of integers.  
	The set of numbers that can be obtained from $0$ by adding or subtracting $1$.
	
	\item $\mathbb{Z}^+ = \mathbb{Z}_{>0} = \{1, 2, 3, 4, 5, \ldots\}$: the set of positive integers.
	
	\item $\mathbb{N} = \mathbb{Z}^+$: the set of natural numbers.  
	(Remark: Some textbooks include $0$ in $\mathbb{N}$.)
	
	\item $\mathbb{Q} = \{\, m/n \mid m,n \in \mathbb{Z},\ n \neq 0 \,\}$: the set of rational numbers.
	
	\item $\varnothing$: the empty set, the set containing no elements.
	
	\item Similarly, we can use $\mathbb{Z}^-$, $\mathbb{Q}^+$, $\mathbb{Q}^-$, $\mathbb{R}^+$, $\mathbb{R}^-$, \dots
	
	\item $\mathbb{C}$ denotes the set of all complex numbers. Not discussed (formally).
	
\end{itemize}

\section{Mathematical Statements}

\subsection{Structures}

\subsubsection{Conditional Statements}

\begin{itemize}
	
	\item \textbf{Standard form:} \emph{If $P$, then $Q$}.
	\item \textbf{Symbolic form:} $P \Rightarrow Q$.
	\item \textbf{Other forms:} \emph{$Q$ is necessary for $P$}, or, \emph{$P$ is sufficient for $Q$}.
	
\end{itemize}

Here $P$ is called the \emph{hypothesis} (or \emph{assumption}), and $Q$ is called the \emph{conclusion}. 
No assumption is made that $P$ is true; rather, the statement asserts that whenever $P$ is true, $Q$ is also true.

\begin{example}
	If $ n > 2 $, then $ n > 1 $.
\end{example}


\subsubsection{Biconditional Statements}

\begin{itemize}
	\item \textbf{Standard form:} \emph{$P$ if and only if $Q$}.
	\item \textbf{Abbreviation:} \emph{$P$ iff $Q$}.
	\item \textbf{Other forms:} \emph{$P$ is necessary and sufficient for $Q$}.
	\item \textbf{Symbolic form:} $P \Leftrightarrow Q$.
	\item \textbf{Meaning:} $P \Rightarrow Q$ and $Q \Rightarrow P$.
\end{itemize}

\begin{example}
	$n$ is odd if and only if $ n + 1 $ is even.
\end{example}

\subsubsection{Existential Statements}

\begin{itemize}
	\item \textbf{Standard form:} \emph{There exists an $x$ in $D$ such that $P(x)$.}
	\item \textbf{Abbreviation:} $\exists\, x \in D,\; P(x)$.
	\item \textbf{$D$ is the domain.}
	\item \textbf{Other forms:} \emph{there is a \dots}, \emph{there exists \dots}, 
	\emph{for some \dots}, \emph{we can find \dots}, \emph{\dots\ has \dots}.
\end{itemize}


\begin{example}
	There exists a real number $x$ such that $ x^2 = 2 $.
\end{example}

\begin{example}
	There is a tallest man in the world.
\end{example}

\subsection{Content}

\subsubsection{Definitions}

\begin{definition}[Definition]
	A mathematical definition is a true mathematical statement that gives the precise meaning of a word or phrase that represents an object, a property, or another concept.
\end{definition}

To state a mathematical statement, we first need precise definitions.
For example, the sentence ``All zaxcillian has rizz'' is meaningless without definitions.

\begin{example}[Negative example]
	From \url{https://www.dictionary.com}:
	\begin{itemize}[
		leftmargin=3em,
		label=\textbullet,
		itemsep=0.2ex,
		topsep=0.2ex,
		parsep=0pt,
		font=\small
		]
		\item Definition of happy: feeling or showing pleasure, contentment, or joy.
		\item Definition of pleased: feeling happy or satisfied.
		\item Examples: winning a match, doing well on an exam, having an ice cream, etc.
	\end{itemize}
\end{example}

\begin{example}
	An integer $a$ is \emph{even} if there exists an integer $n$ such that $a = 2n$.
\end{example}

\begin{example}
	An integer $a$ is \emph{odd} if there exists an integer $n$ such that $a = 2n + 1$.
\end{example}




\subsubsection{Theorems}

\begin{definition}[Theorem]
	A theorem is a true mathematical statement that can be proven mathematically.
\end{definition}

\begin{example}
	$n$ is odd if and only if $ n + 1 $ is even.
\end{example}

\subsubsection{Axioms}

\begin{definition}[Axiom]
	An axiom is a mathematical statement that does not require proof.
\end{definition}

\begin{example}
	If x and y are integers, then so is $ x + y $.
\end{example}

Some axioms of the real numbers $\mathbb{R}$: see Appendix~A for a full list.
\begin{itemize}
	
	\item Closure.  
	For any real numbers $x$ and $y$, $x + y$ and $x \cdot y$ are real numbers.
	
	\item Associative.  
	For any $x, y, z$,
	\[
	x + (y + z) = (x + y) + z \quad \text{and} \quad
	x \cdot (y \cdot z) = (x \cdot y) \cdot z.
	\]
	
	\item Commutative.  
	For any $x$ and $y$,
	\[
	x + y = y + x \quad \text{and} \quad x \cdot y = y \cdot x.
	\]
	
	\item Identity.  
	There exist $0$ and $1$ such that $x + 0 = x$ and $1 \cdot x = x$ for all $x$.
	
	\item Inverse.  
	For any $x$, there exists $-x$ such that $x + (-x) = 0$.  
	If $x \neq 0$, there exists $\frac{1}{x}$ such that $x \cdot \frac{1}{x} = 1$.
	
	\item Distributive.  
	For any $x, y, z$,
	\[
	x \cdot (y + z) = x \cdot y + x \cdot z.
	\]
	
\end{itemize}

It depends on whom you ask. Some authors instead construct the set of real numbers.
See \url{https://en.wikipedia.org/wiki/Construction_of_the_real_numbers}.

Here are some axiomatic systems used in mathematics.
\begin{itemize}
	
	\item Zermelo--Fraenkel (ZF) set theory:  
	\url{https://plato.stanford.edu/entries/set-theory/zf.html}.
	
	\item May include the axiom of choice (ZFC):  
	\url{https://plato.stanford.edu/entries/axiom-choice/}.
	\begin{itemize}
		\item Even with the axiom of choice, mathematicians are still unable to prove the continuum hypothesis.
		\item This led to the notion of independence.
	\end{itemize}
	
	\item Euclid's axioms of geometry:  
	\url{https://plus.maths.org/content/maths-minute-euclids-axioms}.
	\begin{itemize}
		\item The fifth axiom caused difficulties; for example, on a sphere, so-called ``parallel lines'' intersect.
		\item This gives rise to non-Euclidean geometry (manifolds).
	\end{itemize}
	
\end{itemize}


\section{Proofs}

\begin{definition}[Proof]
	A (mathematical) proof is a deductive argument for a mathematical statement, showing that the stated assumptions logically guarantee the conclusion.
\end{definition}
\[
P \xLongrightarrow{\substack{
		\text{definitions, axioms, algebraic manipulations,} \\
		\text{and previously established facts}
}} Q
\]

\begin{example}
	For $a \in \mathbb{Z}$, if $a$ is odd, then $a + 1$ is even.
	
	\begin{itemize}[
		leftmargin=3em,
		label=\textbullet,
		itemsep=0.2ex,
		topsep=0.2ex,
		parsep=0pt,
		font=\small
		]
		
		\item The hypothesis (or assumption) is $P$: $a$ is odd.  
		The conclusion is $Q$: $a + 1$ is even.
		
		\item In short, we want to prove the conditional statement
		\[
		a \text{ is odd} \;\Rightarrow\; a + 1 \text{ is even}.
		\]
		
		\item We may give a direct proof: start with $P$ and end with $Q$.
		
	\end{itemize}
	
\end{example}

\begin{theorem}
	For $a \in \mathbb{Z}$, if $a$ is odd, then $ a + 1 $ is even.
\end{theorem}

\begin{proof}[Not a Proof.]
	3 is odd, 3+1=4 is even.
	
	5 is odd, 5+1=6 is even.
	
	97 is odd, 97+1=98 is even.
	
	etc...
\end{proof}

This is not a proof; it only gives examples. It does not show that the statement is true for all odd integers. 
This is a universal statement (“for any”, “for all”, notated $\forall$). 
Such a statement cannot be proved by exhaustive search.

\begin{proof}[Not a Proof.]
	Suppose we try to find an odd integer $a$ such that $a + 1$ is not even. 
	This is impossible; therefore, the theorem is true.
\end{proof}

The inability to find a counterexample does not constitute a proof. For example, ``All crows are black."

\begin{proof}[Not a Proof.]
	Of course it’s true since my professor said it is. This follows from deep algebraic structures that extend
	beyond elementary divisibility rules, touching on abstract algebra, modular arithmetic, and measure theory.
	Additionally, this result is a trivial corollary of the Generalized Parity Theorem, which is left as an exercise
	to the reader.
\end{proof}

This is called proof by intimidation, but Mathematicians should uphold truth and not be intimidated.

\begin{proof}[(Direct) proof]
	Suppose $a$ is odd. {\small\emph{(Starting from the hypothesis.)}}
	
	There exists an integer $n$ such that $a = 2n + 1$. {\small\emph{(Definition of odd number.)}}
	
	Then $a + 1 = 2n + 2 = 2(n + 1)$. {\small\emph{(Algebraic manipulation.)}}
	
	Since $m = n + 1$ is an integer, {\small\emph{(Use axiom, closure property.)}}
	
	therefore, $a + 2 = 2m$ for some integer $m$. {\small\emph{(Definition of even.)}}
	
	Hence, $a + 1$ is even. {\small\emph{(Conclusion.)}}
\end{proof}

\begin{proof}
	Suppose $a$ is odd. There exists an integer $n$ such that $a = 2n + 1$. Then
	\[
	a + 1 = 2n + 2 = 2(n + 1) = 2m,
	\]
	where $m = n + 1$. Since $m = n + 1$ is an integer, $a + 1 = 2m$ is even.
	
	
\end{proof}

The box in the bottom right indicates the end of the proof. We may also write “QED.”


\subsection{Direct Proof}

A direct proof is an approach to prove a conditional statement $P \Rightarrow Q$.

\begin{itemize}
	\item It is a series of valid arguments that starts with the hypothesis $P$ and ends with the conclusion $Q$.
	\item In the process, we may use definitions, properties, axioms, algebraic manipulations, and other known results.
	\[
	P \;\Rightarrow\; P_1 \;\Rightarrow\; P_2 \;\Rightarrow\; \cdots \;\Rightarrow\; P_k \;\Rightarrow\; Q.
	\]
	\item Keep the conclusion $Q$ in mind at every step; each statement $P_i$ moves closer to $Q$.
	\item We may "cheat" and work backwards from $Q$ a little.
	\item We may need to use existential instantiation.
\end{itemize}

\subsection{Existential Instantiation}

The symbol $\exists$ denotes the existential quantifier and is read as “there exists.”

\begin{itemize}
	
	\item If we know that something exists, we may give it a name (label) and use it.
	
	\item However, do not give two \emph{a priori} different objects the same name (label) in the current discussion.
	
\end{itemize}
	
	\begin{example}
		Let $a$ be an odd integer. Then there exists an integer $n$ such that $a = 2n + 1$.
		\[
		\exists\, n \in \mathbb{Z} \;\text{s.t.}\; a = 2n + 1
		\]
	\end{example}
	
	\begin{example}
		Let $a$ and $b$ be odd integers. Then there exists an integer $n$ such that
		$a = 2n + 1$ and $b = 2n + 1$.
		\begin{itemize}[
			leftmargin=3em,
			label=\textbullet,
			itemsep=0.2ex,
			topsep=0.2ex,
			parsep=0pt,
			font=\small
			]
			\item This would imply that $a = b$, which might not be true.
			\item Instead, there exist integers $n$ and $m$ such that $a = 2n + 1$ and $b = 2m + 1$.
			\[
			\exists\, n,m \in \mathbb{Z} \;\text{s.t.}\; a = 2n + 1,\quad b = 2m + 1
			\]
		\end{itemize}
		
	\end{example}
	
\begin{theorem}
	The product of an even integer with an odd integer is even.
\end{theorem}

\begin{proof}[Not a Proof.]
	Suppose $m$ is even and $n$ is odd. Then
	\[
	mn = 2r = 2(p)(2q + 1) = (2p)(2q + 1)
	\]
	So, let $ m = 2p $ and $ n = 2q + 1 $ show that $ mn = 2r $ is even.
\end{proof}
Question: What is wrong with this proof?

\begin{proof}
	Suppose $m$ is even and $n$ is odd. Then
	\[
	\exists\, p,q \in \mathbb{Z}  \;\text{s.t.}\; m = 2p \;\text{and}\; n = 2q + 1.
	\]
	Therefore,
	\[
	mn = (2p)(2q + 1) = 2(p)(2q + 1) = 2r,
	\]
	where $ r = p(2q + 1) $ is an integer. Hence, $mn$ is even.
\end{proof}

\subsection{Counterexample}

\begin{itemize}
	\item The theorems thus far are universal conditional statements ($\forall$).
	\item To disprove a universal conditional statement, that is, to show that it is false, provide a counter-example.
\end{itemize}

In providing a counter-example, it must fulfil the hypothesis, but not the conclusion. That is, to show that
$P \Rightarrow Q$ is false, 
\[
P (\text{true}) \text{ and } Q(\text{false}) \text{ then } P  Q.
\]

\subsection{Proof by Cases}

\subsection{Proof by Contradiction}

\subsection{Proving Existential Statements}

\subsubsection{Constructive Proof}

\subsubsection{Non-constructive Proof}

\section{Divisibility}

\begin{definition}
	Let $n$ and $d$ be integers such that $d = 0$. We say that $n$ is divisible by $d$, or $d$ divides $n$, denoted as $ d | n $ if and only if there is an integer $k$ such that
	\[
	n = kd
	\]
	Otherwise, we say $d$ does not divide $n$, and denote it as $ d | n $.
\end{definition}

Caution: Not to be confused with $ d / n $

\begin{theorem}
	If $x$ is an even integer, then $x^2$ is divisible by $4$.
\end{theorem}

\begin{exercise}
	Prove the following theorem
		\begin{theorem}
		For any $ a,b,c \in \mathbb{Z}$,
		\[
		a | b \text{ and } a | c \Rightarrow a | (b c)
		\]
		\end{theorem}
\end{exercise}

\section{Congruence Modulo}

\begin{itemize}
	\item In 12hrs format, 7 hours after 10am is 5pm.
	\item In 12hrs format, 5 hours after 8am is 1pm.
	\item In 24hrs format, 12 hours after 1300hrs is 0100hrs the following day.
\end{itemize}

\subsection{Properties of Congruence}

\subsection{Modular Arithmethic}

\section{Truth Table}

Given some (Mathematical) statements, we may form a truth table to list the truth value of some combinations of the statements depending on the truth value of the original statements.

\begin{example}
	Modus Ponen: $ P \Rightarrow Q $
\end{example}

Question: Why is $ P \Rightarrow Q $ true whenever $P$ is false?
This is known as vacously true.

\section{Negation}

	\begin{definition}[Negation]
		Let $P$ be a (mathematical) statement. The negation of $P$ , read as not $P$ , denoted as $P$, is true when $P$ is false, and false when $P$ is true.
	\end{definition}

	\subsection{Negation of Universal Statements}
	
		\begin{itemize}
			\item Recall that universal statements are statements with the quantifier “for all”, “for any”.
		\end{itemize}
		
	\subsection{Negation of Existential Statements}
	
		The negation of an existential statement is a universal statement.

\section{Modus Tollens or Contrapositive}

\section{Sequences and Series}

\section{Pigeonhole Principle}






\end{document}
